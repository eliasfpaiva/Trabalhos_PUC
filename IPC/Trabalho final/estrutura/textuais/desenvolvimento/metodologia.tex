% METODOLOGIA------------------------------------------------------------------

\chapter{METODOLOGIA}
\label{chap:metodologia}
O estudo que será feito é de natureza aplicada analítica e possui abordagem quantitativa e qualitativa em diferentes questões.

\section{DELINEAMENTO DA PESQUISA}
\label{sec:MetDelPesq}
A pesquisa se limita à produção de um indicador capaz de representar a disposição dos indivíduos e instituições consultados em colaborar com a criação do sistema de traduções.

\section{COLETA DE DADOS}
\label{sec:MetColDad}
A coleta de dados será realizada através de um questionário elaborado no Google Formulários (disponível no link: https://goo.gl/forms/MMJOHtpqXLWpit3p2), que apresenta questões diferentes para três perfis de pessoas: Estudante do ensino superior, Representante ou funcionário de uma instituição de ensino superior, Outro perfil. Como cada perfil dentre estes tem uma função diferente em relação à vivência acadêmica e desempenhariam funções diferentes no sistema proposto, caso seja implementado, achou-se conveniente elaborar questões personalizadas para cada perfil consultado.

\section{TRATAMENTO DE DADOS}
\label{sec:MetTratDad}
O Google Formulários apresenta, muito intuitivamente, as respostas de uma pesquisa e também produz uma planilha contento todas as respostas em formato CSV (Valores Separados por Virgula), o que facilita a mineração de outros dados, caso seja necessário. Sobre os resultados obtidos pode-se trabalhar com ferramentas de avaliação estatística disponíveis em linguagens de programação que são fortemente aplicadas a este fim como R e Python.

% INTRODUÇÃO-------------------------------------------------------------------

\chapter{INTRODUÇÃO}
\label{chap:introducao}

É do conhecimento de grande parte da população, que existe uma quantidade enorme de informação disponível no planeta atualmente. Essa massa de informação está disponível para ser consultada de forma fácil, principalmente na internet, mas, há uma barreira que dificulta o uso amplo destas informações e, consequentemente, dificulta a criação e/ou evolução de tecnologias pelo mundo. Esta barreira é a língua.

O intuito final deste trabalho é levar ao desenvolvimento de uma plataforma web que, em parceria com universidades, possa ser um repositório público de conteúdo traduzido. Universitários traduziriam conteúdo em língua estrangeira e professores validariam as traduções para que estas possam ser disponibilizadas para a população. Os alunos tradutores receberiam horas de atividade pelas traduções realizadas, além de absorver conhecimento sobre o conteúdo traduzido e de reforçar sua fluência em outros idiomas, trazendo peso para o currículo. As universidades ficariam reconhecidas pela colaboração oferecida em prol da sociedade reforçando o braço universitário da extensão.

\section{PROBLEMA}
\label{sec:introProblema}
Principalmente em países em desenvolvimento, como o Brasil, o nível de fluência em uma língua estrangeira é muito baixo e isso reduz o acesso a informações do exterior. Atualmente há muitas ferramentas que ajudam na redução deste obstáculo, como o Google Tradutor, mas, existem limitações importantes quando se trata de um linguajar mais técnico que, infelizmente, estes tradutores não conseguem transpor. Isso gera a necessidade de um esforço ainda maior para os pesquisadores que pretendem desenvolver algo com base nestes materiais.

\section{OBJETIVOS}
\label{sec:introObjetivos}
Antes de desenvolver um ambiente como o apresentado anteriormente, é necessário verificar se ele teria adoção da comunidade acadêmica, pois, uma plataforma colaborativa apenas funciona se houver quem colabore com ela. Tendo isto em vista, este trabalho será focado em uma pesquisa para constatar a viabilidade do desenvolvimento do sistema. O desenvolvimento em si é uma proposta de trabalho futuro caso esta pesquisa demonstre que haveria um número de pessoas e instituições interessadas capaz de garantir o funcionamento do sistema.

\section{JUSTIFICATIVA}
\label{sec:introJustificativa}
Apesar de o problema ser real e as dificuldades geradas pela barreira linguística serem evidentes principalmente para os membros corpo discente das universidades, é necessário obter primeiro a indicação de que o projeto será bem sucedido e terá adesão suficiente para se tornar uma boa fonte de informação, do contrário o esforço seria vão.

\section{METAS}
\label{sec:introMetas}
Com o resultado deste estudo buscamos:
\begin{enumerate}
	\item Saber se a comunidade esta disposta a colaborar para que não seja necessário aprender novos idiomas apenas para ter acesso a conteúdo científico de qualidade
	\item Obter argumentos para buscar recursos para o desenvolvimento da plataforma de tradução, caso seja comprovado o interesse no projeto
	\item Facilitar a produção científica tornando menos penosa a tarefa de pesquisa bibliográfica
	\item Elevar a fluência de de alunos e professores universitários em idiomas estrangeiros
	\item Colaborar para a atualização constante dos conhecimentos de alunos e professores universitários
\end{enumerate}
